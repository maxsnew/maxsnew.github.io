\documentclass[12pt]{article}

%AMS-TeX packages
\usepackage{amssymb,amsmath,amsthm} 
%geometry (sets margin) and other useful packages
\usepackage[margin=1.25in]{geometry}
\usepackage{tikz}
\usepackage{tikz-cd}
\usepackage{graphicx,ctable,booktabs}
\usepackage{mathpartir}

\usepackage[sort&compress,square,comma,authoryear]{natbib}
\bibliographystyle{plainnat}

%
%Redefining sections as problems
%
\makeatletter
\newenvironment{problem}{\@startsection
       {section}
       {1}
       {-.2em}
       {-3.5ex plus -1ex minus -.2ex}
       {2.3ex plus .2ex}
       {\pagebreak[3]%forces pagebreak when space is small; use \eject for better results
       \large\bf\noindent{Problem }
       }
       }
       {%\vspace{1ex}\begin{center} \rule{0.3\linewidth}{.3pt}\end{center}}
       \begin{center}\large\bf \ldots\ldots\ldots\end{center}}
\makeatother


%
%Fancy-header package to modify header/page numbering 
%
\usepackage{fancyhdr}
\pagestyle{fancy}
%\addtolength{\headwidth}{\marginparsep} %these change header-rule width
%\addtolength{\headwidth}{\marginparwidth}
\lhead{Problem \thesection}
\chead{} 
\rhead{\thepage} 
\lfoot{\small\scshape EECS 598: Category Theory} 
\cfoot{} 
\rfoot{\footnotesize PS 1} 
\renewcommand{\headrulewidth}{.3pt} 
\renewcommand{\footrulewidth}{.3pt}
\setlength\voffset{-0.25in}
\setlength\textheight{648pt}

%%%%%%%%%%%%%%%%%%%%%%%%%%%%%%%%%%%%%%%%%%%%%%%

%
%Contents of problem set
%

\begin{document}

\newcommand{\skipp}{\textrm{skip}}
\newcommand{\boxprod}{\mathop{\square}}
\newcommand{\Set}{\textrm{Set}}
\newcommand{\Cat}{\textrm{Cat}}
\newcommand{\Cayley}{\textrm{Cayley}}
\newcommand{\Preds}{\mathcal{P}}
\newcommand{\triple}[3]{\{#1\}{#2}\{#3\}}
\newcommand{\Triple}{\textrm{Triple}}
\newcommand{\Analyse}{\textrm{Analyse}}
\newcommand{\command}{\textrm{command}}

\newcommand{\id}{\textrm{id}}

\newcommand{\isatype}{\,\,\textrm{type}}
\newcommand{\matchZero}{\textrm{match}_0}
\newcommand{\matchSum}[3]{\textrm{match}_+ {#1}\{{#2}\}\{{#3}\}}

\title{Problem Set 4}
\date{February 7, 2022}
\maketitle

Homework is due the midnight before class on the 17th.
%
If you want to volunteer to present this problem, e-mail me.

Both problems involve the free bi-cartesian category (BiCC) generated
from an empty graph. We present this in full here for reference.

First, the types
\begin{mathpar}
  \inferrule{}{1 \isatype}
  
  \inferrule{A \isatype \and B \isatype}{A \times B \isatype}

  \inferrule{}{0 \isatype}

  \inferrule{A \isatype \and B \isatype}{A + B \isatype}
\end{mathpar}

Next, the terms
\begin{mathpar}
  \inferrule
  {}
  {x : A\vdash () : 1}

  \inferrule
  {x:A \vdash t_1 : B_1 \and
   x:A \vdash t_2 : B_2
  }
  {x:A \vdash (t_1,t_2) : B_1 \times B_2 }

  \inferrule
  {x:A \vdash t : B_1 \times B_2}
  {x:A \vdash \pi_1 t : B_1}

  \inferrule
  {x:A \vdash t : B_1 \times B_2}
  {x:A \vdash \pi_2 t : B_2}

  \inferrule
  {}
  {x:0 \vdash \matchZero x : A }

  \inferrule
  {x : A \vdash t : A_1 + A_2 \and
   x_1:A_1 \vdash t_1 : B \and
   x_2:A_2 \vdash t_2 : B
  }
  {x:A \vdash \matchSum t {\sigma_1 x_1. t_1}{\sigma_2 x_2. t_2} : B}

  \inferrule
  {x:A \vdash t : B_1}
  {x:A \vdash \sigma_1 t : B_1 + B_2}

  \inferrule
  {x:A \vdash t : B_2}
  {x:A \vdash \sigma_2 t : B_1 + B_2}
\end{mathpar}

And the definition of substitution $t[s/x]$ by induction on $t$:
\begin{align*}
  x[s/x] &= s\\
  ()[s/x] &= ()\\
  (t_1,t_2)[s/x] &= (t_1[s/x], t_2[s/x])\\
  (\pi_i t)[s/x] &= \pi_i t[s/x]\\
  (\matchZero t)[s/x] &= \matchZero {t[s/x]}\\
  (\matchSum t {\sigma_1 x_1. u_1}{\sigma_2 x_2. u_2})[s/x]  &= \matchSum {t[s/x]} {\sigma_1 x_1. u_1}{\sigma_2 x_2. u_2}\\
  (\sigma_i t)[s/x] &= \sigma_i t[s/x]\\
\end{align*}

And finally the equational theory. Note that we implicitly assume in
any equality $s = t$ that both are well-typed with the same typing.
\begin{mathpar}
  \inferrule
  {}
  {t = t}

  \inferrule
  {s = t}
  {t = s}

  \inferrule
  {s = t \and t = u}
  {s = u}

  \inferrule
  {t = t'}
  {\pi_i t = \pi_i t'}

  \inferrule
  {t_1 = t_1' \and t_2 = t_2'}
  {(t_1,t_2) = (t_1', t_2')}

  \inferrule
  {t = t'}
  {\sigma_i t = \sigma_i t'}

  \inferrule
  {t = t' \and s_1 = s_1' \and s_2 = s_2'}
  {\matchSum t {\sigma_1 x_1. s_1}{\sigma_2 x_2. s_2} = \matchSum {t'} {\sigma_1 x_1. s_1'}{\sigma_2 x_2. s_2'}}

  \inferrule
  {x:A \vdash t : 1}
  {t = ()}
  
  \pi_i(t_1,t_2) = t_i

  \inferrule
  {x:A \vdash t : B_1 \times B_2}
  {t = (\pi_1 t, \pi_2 t)}

  \inferrule
  {x:A \vdash s : 0 \and y: 0 \vdash t : B}
  {t[s/y] = \matchZero s}

  \matchSum {\sigma_i t}{\sigma_1 x_1. s_1}{\sigma_2 x_2. s_2} = s_i[t/x_i]
  
  \inferrule
  {x:A \vdash s : A_1 + A_2 \and
   y: A_1 + A_2 \vdash t : B}
  {t[s/x] = \matchSum s {\sigma_1 x_1. t[\sigma_1x_1/y]}{\sigma_2 x_2. t[\sigma_2x_2/y]}}
\end{mathpar}

In class we showed that this presents a category FreeBiCC that is
initial in the category of bi-cartesian categories and bi-cartesian
functors, i.e., functors that preserve bi-cartesian structure.

\begin{problem}{Duality}
  Notice that every type in the free BiCC has a dual: products are
  dual to coproducts and terminal objects are dual to initial objects.

  To demonstrate this duality, define an \emph{op} translation from
  the free BiCC to its opposite as follows.
  \begin{enumerate}
  \item Define a translation $A^{op}$ for each type.
  \item Define a translation $t^{op}$ on terms such that if
    \[ x : A \vdash t : B \]
    then
    \[ k: B^{op} \vdash t^{op} A^{op} \]
  \item Show that the translation is an involution: $(A^{op})^{op} = A$
    and for any $t$, $(t^{op})^{op} = t$ in the equational theory. Are
    there any $t$ such that $(t^{op})^{op}$ is not syntactically
    identical to $t$? If so, provide an example.
  \end{enumerate}
\end{problem}

\begin{problem}{Logical Relations}
  We can interpret the terms of the free Bi-cartesian category on an
  empty graph as a very simple programming language or formal logic.

  As a logic, we interpet $0$ as falsehood, $+$ as disjunction, $1$ as
  trivial truth and $\times$ as conjunction. To be a useful logic, we
  should verify that there is no proof of false. Of course, there are
  proofs such as
  \[ x : 0 \times (1 + 1) \vdash \pi_1 x : 0 \]
  but we would like that there is no proof of false from a trivial
  assumption, i.e., there is no proof
  \[ x:1 \vdash t : 0 \]
  This property is called \emph{consistency} of a logic.

  To be a useful programming language\footnote{or \emph{constructive}
  logic}, we should verify that we can \emph{evaluate} programs down
  to a result. Since we added no facilities for effects or recursion,
  all programs should run to some kind of value in finite time. To be
  concrete, we should prove that any program of type $1 + 1$ evaluates
  to $\sigma_1()$ or $\sigma_2()$, which we can think of as a
  boolean. The specification for our evaluator is that it should
  respect our notion of $\beta\eta$ equality, so $t$ should evaluate
  to $\sigma_1()$ if and only if $t = \sigma_1()$ in our equational
  theory. Of course, there are terms of type $1 + 1$ that are not
  equal to $\sigma_1()$ or $\sigma_2()$, such as
  \[ x : 1 + 1 \vdash x : 1 + 1 \]
  instead we should restrict to programs that only take a trivial
  input. So we want to show that terms of the form:
  \[ x : 1 \vdash t : 1 + 1 \]
  Are all equal to $\sigma_1()$ or $\sigma_2()$.  A programming
  languages that satisfies this or a similar property is called
  \emph{normalizing}.

  We should also make sure that our programming language specification
  is not trivial. In particular, since we did not add facilities for
  non-determinism, we should verify that if a program evaluates to
  $\sigma_1()$, then it cannot also evaluate to $\sigma_2()$. That is,
  we should show that $\sigma_1() \neq \sigma_2()$.

  \begin{itemize}
  \item Show that the terms $x : 1 \vdash \sigma_1() : 1 + 1$ and
    $x:1\vdash \sigma_2() : 1 + 1$ in the free BiCC on the empty graph
    are not equal. Hint: use the UMP of the free BiCC to construct a
    functor $F$ from the free BiCC to another category where
    $F(\sigma_1()) \neq F(\sigma_2())$.
  \end{itemize}

  Showing consistency/normalization for a calculus can be
  difficult. One of the most robust proof techniques for proving this
  is called the method of \emph{logical relations}. Let's use the
  method of logical relations to prove consistency and normalization
  of the free BiCC.

  The method of logical relations in this case is centered on a
  category we call \emph{LR} defined as folows:
  \begin{enumerate}
  \item The objects of LR are pairs $(A, P)$ of a type $A$ in the free
    BiCC and a subset of the terms of type $A$ with trivial input: $P
    \subseteq \{ t \,|\, x:1 \vdash t : A \}$
  \item A morphism from $(A, P)$ to $(B, Q)$ is a term $y : A \vdash s
    : B$ such that for any $t \in P$, $s[t/y] \in B$.
  \end{enumerate}

  We will then show that LR is itself a bi-cartesian category with the
  following definitions of the objects involved:
  \begin{enumerate}
  \item The initial object is $(0, \emptyset)$
  \item The coproduct of $(A, P)$ and $(B, Q)$ is $(A + B, R_+)$ where $R_+$ is
    \[ \{ u \,|\, (\exists t \in P.~ u = \sigma_1 t) \vee (\exists t \in Q.~ u = \sigma_2 t) \} \]
  \item The terminal object is $(1, \top)$ where $\top = \{ t \,|\,
    x:1 \vdash t : 1 \}$.
  \item The product of $(A, P)$ and $(B, Q)$ is $(A \times B, R_\times)$ where $R_\times$ is
    \[ \{ u \,|\, (x : 1 \vdash u : A \times B) \wedge \pi_1 u \in P \wedge \pi_2 u \in Q \} \]
  \end{enumerate}

  Now, prove consistency and normalization for BiCC as follows:
  \begin{enumerate}
  \item Define identity and composition in LR, and prove it forms a
    category.
  \item Show that the definitions above give LR the structure of a
    bi-cartesian category, and that the ``forgetful'' functor $U :
    \textrm{LR} \to \textrm{FreeBiCC}$ preserves the bi-cartesian
    structure, i.e., it takes products to products, projections to
    projections, etc.
  \item Conclude using the UMP of the free bi-cartesian category that
    there is a functor from $s : \textrm{FreeBiCC} \to \textrm{LR}$
    that preserves bi-cartesian structure and is a section of $U$,
    that is:
    \[U \circ s = \textrm{id}_{FreeBiCC}\]

  \item Prove as a corollary that
    \begin{enumerate}
    \item BiCC is consistent: there is no term $x : 1 \vdash t : 0$ 
    \item BiCC is normalizing: any term $x : 1 \vdash t : 1 + 1$ is
      equal to $\sigma_1()$ or $\sigma_2()$ in the equational theory.
    \end{enumerate}
  \end{enumerate}
\end{problem}

\thispagestyle{empty}


\bibliography{cats}
\end{document}
