\documentclass[12pt]{article}

%AMS-TeX packages
\usepackage{amssymb,amsmath,amsthm} 
%geometry (sets margin) and other useful packages
\usepackage[margin=1.25in]{geometry}
\usepackage{tikz}
\usepackage{tikz-cd}
\usepackage{graphicx,ctable,booktabs}

\usepackage[sort&compress,square,comma,authoryear]{natbib}
\bibliographystyle{plainnat}

%
%Redefining sections as problems
%
\makeatletter
\newenvironment{problem}{\@startsection
       {section}
       {1}
       {-.2em}
       {-3.5ex plus -1ex minus -.2ex}
       {2.3ex plus .2ex}
       {\pagebreak[3]%forces pagebreak when space is small; use \eject for better results
       \large\bf\noindent{Problem }
       }
       }
       {%\vspace{1ex}\begin{center} \rule{0.3\linewidth}{.3pt}\end{center}}
       \begin{center}\large\bf \ldots\ldots\ldots\end{center}}
\makeatother


%
%Fancy-header package to modify header/page numbering 
%
\usepackage{fancyhdr}
\pagestyle{fancy}
%\addtolength{\headwidth}{\marginparsep} %these change header-rule width
%\addtolength{\headwidth}{\marginparwidth}
\lhead{Problem \thesection}
\chead{} 
\rhead{\thepage} 
\lfoot{\small\scshape EECS 598: Category Theory} 
\cfoot{} 
\rfoot{\footnotesize PS 1} 
\renewcommand{\headrulewidth}{.3pt} 
\renewcommand{\footrulewidth}{.3pt}
\setlength\voffset{-0.25in}
\setlength\textheight{648pt}

%%%%%%%%%%%%%%%%%%%%%%%%%%%%%%%%%%%%%%%%%%%%%%%

%
%Contents of problem set
%

\begin{document}

\newcommand{\skipp}{\textrm{skip}}
\newcommand{\boxprod}{\mathop{\square}}
\newcommand{\Set}{\textrm{Set}}
\newcommand{\Cat}{\textrm{Cat}}
\newcommand{\Cayley}{\textrm{Cayley}}
\newcommand{\Preds}{\mathcal{P}}
\newcommand{\triple}[3]{\{#1\}{#2}\{#3\}}
\newcommand{\Triple}{\textrm{Triple}}
\newcommand{\Analyse}{\textrm{Analyse}}
\newcommand{\command}{\textrm{command}}

\newcommand{\id}{\textrm{id}}

\title{Problem Set 3}
\date{January 27, 2022}
\maketitle

%% Homework is due midnight before next Thursday's class. Starred
%% assignments will be presented in class. Turn in your homework on
%% Canvas. You may work alone or in groups of two. I highly encourage you
%% to write your solution in LaTeX. The LaTeX packages tikz and tikz-cd
%% are useful for typesetting commutative diagrams, and there are WYSIWYG
%% editors online that produce LaTeX code such as https://q.uiver.app/
%% and https://tikzcd.yichuanshen.de/ .

\thispagestyle{empty}


\begin{problem}{The Product of Graphs}
  When working in a new mathematical domain, we can use category
  theoretic concepts to help us understand common constructions.

  For this problem, a graph $G = (G_v, G_e)$ consists of a set of
  vertices $G_v$ and an incidence relation $G_e \subseteq G_v^2$
  telling us when there is a (directed) edge between two vertices.

  A graph homomorphism $\phi : G \to H$ is a function on the vertices
  $\phi : G_v \to H_v$ that preserves the incidence relation: if
  $(v,v')\in G_e$ then $(\phi(v), \phi(v')) \in H_e$.

  There are many constructions that are called the ``product'' of
  graphs, but up to isomorphism only one that satisfies the universal
  mapping property of a product.

  Consider the following two constructions.

  For any two graphs $G, H$ define the \emph{box product} $G \boxprod
  H$ to be the following graph:
  \begin{enumerate}
  \item The set of vertices is given by the cartesian product
    $(G\boxprod H)_v = G_v \times H_v$.
  \item There is an edge $((g,h), (g',h')) \in (G\boxprod H)_e$ if and
    only if either of the following holds
    \begin{enumerate}
    \item $(g,g') \in G_e$ and $h = h'$.
    \item $g = g'$ and $(h,h') \in H_e$.
    \end{enumerate}
  \end{enumerate}

  For any two graphs $G,H$ define the \emph{tensor product} $G\otimes
  H$ to be the following graph:
  \begin{enumerate}
  \item The set of vertices is given by the cartesian product
    $(G\otimes H)_v = G_v \times H_v$.
  \item There is an edge $((g,h), (g',h')) \in (G\otimes H)_e$ if and
    only if $(g,g') \in G_e$ and $(h,h') \in H_e$.
  \end{enumerate}

  Both of these are useful ways to combine two graphs together, but
  only one of them is the true categorical product.
  \begin{enumerate}
  \item Show that one of these two products ($\boxprod$, $\otimes$)
    gives the categorical product of graphs, i.e., always satisfies
    the universal mapping property of a product.
  \item For the other, briefly explain why it fails to be a
    categorical product.
  \end{enumerate}
\end{problem}

\begin{problem}{}
  When programming we often have an implicit \emph{invariant} on our
  datatypes that is too complex to encode in our type system, but is
  crucial for justifying the correctness or safety of the code.

  This can be modeled by a category of \emph{subsets} where an object
  $(X,P)$ is a pair of a set $X$ and a subset $P \subseteq X$ of those
  values that satisfy the invariant. A morphism $f : (X,P) \to (Y,Q)$
  is a function $f : X\to Y$ that preserves the invariant: if $p \in
  P$ then $f(p) \in Q$.

  We can generalize this from subsets to ``sub-objects'' in any
  category as follows. Let $\mathbb C$ be a category. A sub-object of
  an object $A$ is a monomorphism $m : P \rightarrowtail A$. A
  morphism of sub-objects $(f, \phi) : (m : P \rightarrowtail A) \to
  (n : Q \rightarrowtail B)$ consists of a morphism $f : A \to B$ and
  a morphism $\phi : P \to Q$ such that the following diagram
  commutes:

  \[% https://tikzcd.yichuanshen.de/#N4Igdg9gJgpgziAXAbVABwnAlgFyxMJZABgBpiBdUkANwEMAbAVxiRAAUQBfU9TXfIRQBGclVqMWbAIrdeIDNjwEiZYePrNWiEAEE5fJYKKj11TVJ0AhbuJhQA5vCKgAZgCcIAWyRkQOCCQAJmocOiwGNjCIkHNJbRAfHjdPH0RRf0DEAGZQ8MidaMi4rTZCZJAPb19QrIyLBIAdRrQACywDStTg2qRciVKdV1suIA
  \begin{tikzcd}
    P \arrow[d, "m", tail] \arrow[r, "\phi"] & Q \arrow[d, "n", tail] \\
    A \arrow[r, "f"]                         & B                     
  \end{tikzcd}\]

  The defines for any category $\mathbb C$ a category
  $\textrm{Sub}(\mathbb C)$ of subobjects. Composition and identity are
  defined the same as in the arrow category (Awodey Chapter 1.6).

  For $\mathbb C = \Set$, $\textrm{Sub}(\Set)$ this generalizes our
  category of subsets slightly to allow arbitrary injective functions
  $m : P \rightarrowtail A$ as objects. We can view such a function as
  a presentation of the subset that is its image $\{ x \in A | \exists
  p \in P. m(p) = x \}$. Then the property about $\phi$ demonstrates
  that $f$ preserve this subset.

  Let $\mathbb C$ be any category with an initial object $0$, terminal
  object $1$ and for any pair of objects $A,B \in \mathbb C$ a product
  $A \times B$.

  \begin{itemize}
  \item
    Prove that $\textrm{Sub}(\mathbb C)$ has an initial object, terminal
    object and all (binary) products.
  \end{itemize}
\end{problem}

\bibliography{cats}
\end{document}
