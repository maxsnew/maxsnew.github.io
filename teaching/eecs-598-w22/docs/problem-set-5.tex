\documentclass[12pt]{article}

%AMS-TeX packages
\usepackage{amssymb,amsmath,amsthm} 
%geometry (sets margin) and other useful packages
\usepackage[margin=1.25in]{geometry}
\usepackage{tikz}
\usepackage{tikz-cd}
\usepackage{graphicx,ctable,booktabs}
\usepackage{mathpartir}

\usepackage[sort&compress,square,comma,authoryear]{natbib}
\bibliographystyle{plainnat}

%
%Redefining sections as problems
%
\makeatletter
\newtheorem{lemma}{Lemma}
\newenvironment{problem}{\@startsection
       {section}
       {1}
       {-.2em}
       {-3.5ex plus -1ex minus -.2ex}
       {2.3ex plus .2ex}
       {\pagebreak[3]%forces pagebreak when space is small; use \eject for better results
       \large\bf\noindent{Problem }
       }
       }
       {%\vspace{1ex}\begin{center} \rule{0.3\linewidth}{.3pt}\end{center}}
       \begin{center}\large\bf \ldots\ldots\ldots\end{center}}
\makeatother


%
%Fancy-header package to modify header/page numbering 
%
\usepackage{fancyhdr}
\pagestyle{fancy}
%\addtolength{\headwidth}{\marginparsep} %these change header-rule width
%\addtolength{\headwidth}{\marginparwidth}
\lhead{Problem \thesection}
\chead{} 
\rhead{\thepage} 
\lfoot{\small\scshape EECS 598: Category Theory} 
\cfoot{} 
\rfoot{\footnotesize PS 1} 
\renewcommand{\headrulewidth}{.3pt} 
\renewcommand{\footrulewidth}{.3pt}
\setlength\voffset{-0.25in}
\setlength\textheight{648pt}

%%%%%%%%%%%%%%%%%%%%%%%%%%%%%%%%%%%%%%%%%%%%%%%

%
%Contents of problem set
%

\begin{document}

\newcommand{\skipp}{\textrm{skip}}
\newcommand{\boxprod}{\mathop{\square}}
\newcommand{\Set}{\textrm{Set}}
\newcommand{\Cat}{\textrm{Cat}}
\newcommand{\Cayley}{\textrm{Cayley}}
\newcommand{\Preds}{\mathcal{P}}
\newcommand{\triple}[3]{\{#1\}{#2}\{#3\}}
\newcommand{\Triple}{\textrm{Triple}}
\newcommand{\Analyse}{\textrm{Analyse}}
\newcommand{\command}{\textrm{command}}

\newcommand{\id}{\textrm{id}}

\newcommand{\isatype}{\,\,\textrm{type}}
\newcommand{\matchZero}{\textrm{match}_0}
\newcommand{\matchSum}[3]{\textrm{match}_+ {#1}\{{#2}\}\{{#3}\}}

\title{Problem Set 5}
\date{Mar 9, 2022}
\maketitle

Homework is due the midnight before class on March 17.

For these two problems, you are asked to construct morphisms in a
cartesian closed category and prove they satisfy certain
properties. To construct a morphism $A \to B$, you can either
\begin{enumerate}
\item Use the abstract category operations in Awodey such as
  $\epsilon, \tilde f, \pi_i, (f,g)$ etc.
\item Define a term $x : A \vdash t : B$ in the simple type theory
  described in Section 2.8 of Shulman. Then you need to also
  appropriately interpret composition as substitution, etc. In
  Shulman, the term syntax is on page 135 $\beta\eta$ equations for
  $0,+,1,\times$ are on page 114 and $\beta\eta$ for $\Rightarrow$ are
  on page 139.
\end{enumerate}

\begin{problem}{Algebra of Exponentials}
  At this point we have developed quite the ``algebra'' of objects in
  cartesian closed categories: in addition to addition $A + B, 0$ and
  multiplication $A \times B, 1$ we have now added exponentiation
  $B^A$.

  In a cartesian closed category, many of the familiar rules of
  arithmetic with exponentiation are valid if we represent equality as
  \emph{isomorphism}:
  \begin{mathpar}
    C^{A \times B} \cong (C^{A})^B\and
    C^{1} \cong C\\
    C^{A + B} \cong C^{A}\times C^B\and
    C^{0} \cong 1\and
  \end{mathpar}

  \begin{enumerate}
  \item For each of the above isomorphisms $A \cong B$, construct
    morphisms $\textrm{to}:A \to B$ and $\textrm{fro}:B \to A$ that are valid in an arbitrary
    cartesian closed category.
  \item Prove that the morphisms you defined for $C^{A+B} \cong C^A
    \times C^B$ are an isomorphism, that is $\textrm{to} \circ
    \textrm{fro} = \id_B$ and $\textrm{fro} \circ \textrm{to} =
    \id_A$. You do not need to prove the others form an
    isomorphism\footnote{but it will be impossible to construct
    morphisms of the given type that \emph{aren't} isomorphisms!}
  \end{enumerate}
\end{problem}

\begin{problem}{Lawvere's Fixed Point Theorem and Recursive Functions}
  There are various ``diagonalization arguments'' employed in logic,
  set theory and computer science such as the proofs of G\"odel's
  incompleteness theorem, Cantor's theorem and Turing/Rice's theorem
  to prove that some construction is impossible.

  F. William Lawvere showed that we can abstract over the reasoning in
  these different proofs by proving a \emph{fixed point} theorem that
  is valid in an arbitrary cartesian closed category.  We will prove
  the following slightly simplified version of \emph{Lawvere's Fixed
  Point Theorem}.

  Let $X, D$ be objects in a cartesian closed category. If there is a
  section-retraction pair $(s : D^X \to X, r : X \to D^X)$ then there
  is a morphism $\textrm{fix} : D^D \to D$ that is a \emph{fixed point
  combinator} in that for any $f : \Gamma \to D^D$,
  \[ \textrm{fix} \circ f = \epsilon\circ(f, \textrm{fix} \circ f) \]

  In type theoretic notation, this is a term
  \[ f: (D \Rightarrow D) \vdash \textrm{fix}(f) : D \]
  satisfying
  \[ \textrm{fix}(f) = f(\textrm{fix}(f)) \]

  \begin{enumerate}
  \item Instantiate the theorem to the category of sets with $D =
    \{\textrm{true}, \textrm{false} \}$ and $X$ an arbitrary set to
    prove \emph{Cantor's theorem}: there is no surjective function
    from $X$ to its power set $\mathcal P(X)$.

  \item Given the section-retraction pair $(s,r)$, construct
    $\textrm{fix}$ and prove it is a fixed point combinator.
  \end{enumerate}

  This theorem is not just for proving contradictions! You may
  recognize $\textrm{fix}$ as a typed version of the Y-combinator,
  which can be used to encode recursion in programming languages.

  As an example, given a base type $N$ for natural numbers and
  function symbols\footnote{We interpret \textrm{sub1}(n) as returning $\sigma_1()$ if $n=0$ and otherwise $\sigma_2(n - 1)$} $\textrm{sub1} : N \to 1 + N$ and $\textrm{times} : N, N \to N$
  and $\textrm{one} : \cdot \to N$, we can use our fixpoint combinator
  to implement a factorial function:
  \[ \textrm{fix}(\lambda \textrm{fac}.\lambda n. \matchSum{(\textrm{sub1}(n))}{\sigma_1x_1. \textrm{one}}{ \sigma_2 m. \textrm{times}(n,\textrm{fac}\,m)} \]
\end{problem}
\thispagestyle{empty}

\bibliography{cats}
\end{document}
