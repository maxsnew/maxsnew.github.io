\documentclass[12pt]{article}

%AMS-TeX packages
\usepackage{amssymb,amsmath,amsthm} 
%geometry (sets margin) and other useful packages
\usepackage[margin=1.25in]{geometry}
\usepackage{tikz}
\usepackage{tikz-cd}
\usepackage{graphicx,ctable,booktabs}
\usepackage{mathpartir}

\usepackage[sort&compress,square,comma,authoryear]{natbib}
\bibliographystyle{plainnat}

%
%Redefining sections as problems
%
\makeatletter
\newtheorem{lemma}{Lemma}
\newenvironment{problem}{\@startsection
       {section}
       {1}
       {-.2em}
       {-3.5ex plus -1ex minus -.2ex}
       {2.3ex plus .2ex}
       {\pagebreak[3]%forces pagebreak when space is small; use \eject for better results
       \large\bf\noindent{Problem }
       }
       }
       {%\vspace{1ex}\begin{center} \rule{0.3\linewidth}{.3pt}\end{center}}
       \begin{center}\large\bf \ldots\ldots\ldots\end{center}}
\makeatother


%
%Fancy-header package to modify header/page numbering 
%
\usepackage{fancyhdr}
\pagestyle{fancy}
%\addtolength{\headwidth}{\marginparsep} %these change header-rule width
%\addtolength{\headwidth}{\marginparwidth}
\lhead{Problem \thesection}
\chead{} 
\rhead{\thepage} 
\lfoot{\small\scshape EECS 598: Category Theory} 
\cfoot{} 
\rfoot{\footnotesize PS 1} 
\renewcommand{\headrulewidth}{.3pt} 
\renewcommand{\footrulewidth}{.3pt}
\setlength\voffset{-0.25in}
\setlength\textheight{648pt}

%%%%%%%%%%%%%%%%%%%%%%%%%%%%%%%%%%%%%%%%%%%%%%%

%
%Contents of problem set
%

\begin{document}

\newcommand{\skipp}{\textrm{skip}}
\newcommand{\boxprod}{\mathop{\square}}
\newcommand{\Set}{\textrm{Set}}
\newcommand{\Cat}{\textrm{Cat}}
\newcommand{\Cayley}{\textrm{Cayley}}
\newcommand{\Preds}{\mathcal{P}}
\newcommand{\triple}[3]{\{#1\}{#2}\{#3\}}
\newcommand{\Triple}{\textrm{Triple}}
\newcommand{\Analyse}{\textrm{Analyse}}
\newcommand{\command}{\textrm{command}}

\newcommand{\id}{\textrm{id}}
\newcommand{\cat}{\mathbb}

\newcommand{\isatype}{\,\,\textrm{type}}
\newcommand{\matchZero}{\textrm{match}_0}
\newcommand{\matchSum}[3]{\textrm{match}_+ {#1}\{{#2}\}\{{#3}\}}

\title{Problem Set 6}
\date{Mar 20, 2022}
\maketitle

Homework is due the midnight before class on March 29.

\begin{problem}{2-dimensional Category Theory}
  We showed in class that natural transformations form a category with
  ``sequential composition''. That is, if $\alpha : F \Rightarrow F'$
  and $\alpha' : F' \Rightarrow F''$ then $\alpha' \circ \alpha : F
  \Rightarrow F''$ and this composition is associative and unital,
  forming the morphisms of the category of functors.

  There is another way to compose natural transformations, called the
  \emph{Godemont product} or the \emph{parallel} composition.

  For the following, let $\cat B, \cat C,\cat D, \cat E$ be
  categories, $F,G,H,K,I,J$ be functors and natural transformations
  $\alpha,\beta,\gamma$ be natural transformations with domains and
  codomains given by the following diagram:
  \[% https://tikzcd.yichuanshen.de/#N4Igdg9gJgpgziAXAbVABwnAlgFyxMJZAJgBoAGAXVJADcBDAGwFcYkQAdDgW3pwAsARoIAEAYRABfUuky58hFABYK1Ok1bsuvAcJEARKTJAZseAkQBsqmgxZtEnHnyGiAokdlmFRAMw31ey0OQRgcei59CAB3MHoAJ3iYzxM5c0VkAFYAu01HLgBzem5eSJi4xOTpL3kLFHIcjQcnHVcRACEU01qMgEZGoPyOJjR+CI4o2ISk6Kk1GCgC+CJQADMk7iQGkBwIJH6QULAoJF9t3OaACRT1iE3Ebd39miOTxABaM9sm9gBJEBojHooUYAAU0j5HPEsAV+Dgbhtnjs9ogyIcYMdTucfo4AFIIu5Ip6ol4Yt6fbGDEAAaQBICBIPB3jqIGhsPh1RAt3u2WRW1JmMQX0CeRAADECTyaMTtq8kBTvlSAOJ0hkwMEQllsuFzSRAA
\begin{tikzcd}
\mathbb B \arrow[rr, "F", bend left] \arrow[rr, "G"', bend right] & \alpha\Downarrow & \mathbb C \arrow[rr, "H", bend left] \arrow[rr, "I"', bend right] & \beta\Downarrow & \mathbb D \arrow[rr, "J", bend left] \arrow[rr, "K"', bend right] & \gamma\Downarrow & \mathbb E
\end{tikzcd}\]

\begin{enumerate}
\item Show that for any such $\alpha,\beta$ we can define the
  \emph{parallel composition} $\beta \odot \alpha : H \circ F
  \Rightarrow I \circ G$.

  \[ % https://tikzcd.yichuanshen.de/#N4Igdg9gJgpgziAXAbVABwnAlgFyxMJZABgBpiBdUkANwEMAbAVxiRAB12BbOnACwBGAgAQAhEAF9S6TLnyEUAJnJVajFm049+Q4QBFJ0kBmx4CRAIwrq9Zq0Qd2AmDjrDO0CDnftGaPm6cehAA7mB0AE4RoZKqMFAA5vBEoABm0VxIZCA4EEhWIM5gUEgALACcNur2IAASPgDGWBENwgBihmkZWdS5+dRFJYgAtBVVdmwAko3NrQDiINQMdM4MAAqyZgogEVgJfDixEkA
\begin{tikzcd}
\mathbb B \arrow[rr, "H \circ F", bend left=49] \arrow[rr, "I \circ G"', bend right=49] & \beta \odot \alpha \Downarrow & \mathbb D
\end{tikzcd} \]
\item Prove that the parallel composition is associative: $\gamma
  \odot (\beta \odot \alpha) = (\gamma \odot \beta)\odot \alpha$
\item Prove that parallel composition is unital: $\alpha \odot \id_{\id_{\cat B}} = \alpha = \id_{\id_{\cat B}} \odot \alpha$
\item Prove the \emph{interchange law} with the sequential composition
  holds: \[(\beta' \circ \beta) \odot (\alpha' \circ \alpha) = (\beta'
  \odot \alpha') \circ (\beta \odot \alpha)\] where $\beta' : I
  \Rightarrow I'$ and $\alpha' : G \Rightarrow G'$.
\end{enumerate}
\end{problem}

\begin{problem}{Theorems for Free, Naturally}
  In a pure functional language without any reflection mechanisms, all
  definable polymorphic functions are natural. Phil Wadler, building
  on John Reynold's theory of parametricity, called this idea
  ``theorems for free'': just from the type of a polymorphic function,
  the naturality property\footnote{more generally, dinaturality or
  more generally still, parametricity} gives you properties that hold
  for every function of that type\citep{reynolds83,
    wadler1989theorems}.

  For instance, given any function of the type $\forall X. X \to X$,
  i.e., a polymorphic function from $X$ to $X$, generic in $X$, it can
  be shown that the function must be equivalent to the identity
  function. In effectful languages we need to weaken this
  statement. For instance in a language with recursive functions (and
  thus infinite loops) but no other effects, the only functions of
  type $\forall X. X \to X$ are the identity function and the function
  that always loops.

  Your task is to prove these free theorems are true as consequences
  of naturality in different categories that model languages with
  effects. Note that we can model a polymorphic function $\forall X. X
  \to X$ as a natural transformation from $\id_{\cat C}$ to $\id_{\cat
    C}$ where $\cat C$ is the category modeling the functions in our
  programming language.
  \begin{enumerate}
  \item First, prove that the only natural transformation from
    $\id_{\Set}$ to $\id_{\Set}$ is the identity transformation.
  \item Let $\textrm{Par}$ be the category of sets and \emph{partial}
    functions. Prove that the only natural transformations from
    $\id_{\textrm{Par}}$ to $\id_{\textrm{Par}}$ are the identity
    transformation and the transformation that is everywhere
    undefined.
  \end{enumerate}
\end{problem}

\thispagestyle{empty}

\bibliography{cats}
\end{document}
