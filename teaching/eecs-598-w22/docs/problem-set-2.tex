\documentclass[12pt]{article}

%AMS-TeX packages
\usepackage{amssymb,amsmath,amsthm} 
%geometry (sets margin) and other useful packages
\usepackage[margin=1.25in]{geometry}
\usepackage{tikz}
\usepackage{tikz-cd}
\usepackage{graphicx,ctable,booktabs}

\usepackage[sort&compress,square,comma,authoryear]{natbib}
\bibliographystyle{plainnat}

%
%Redefining sections as problems
%
\makeatletter
\newenvironment{problem}{\@startsection
       {section}
       {1}
       {-.2em}
       {-3.5ex plus -1ex minus -.2ex}
       {2.3ex plus .2ex}
       {\pagebreak[3]%forces pagebreak when space is small; use \eject for better results
       \large\bf\noindent{Problem }
       }
       }
       {%\vspace{1ex}\begin{center} \rule{0.3\linewidth}{.3pt}\end{center}}
       \begin{center}\large\bf \ldots\ldots\ldots\end{center}}
\makeatother


%
%Fancy-header package to modify header/page numbering 
%
\usepackage{fancyhdr}
\pagestyle{fancy}
%\addtolength{\headwidth}{\marginparsep} %these change header-rule width
%\addtolength{\headwidth}{\marginparwidth}
\lhead{Problem \thesection}
\chead{} 
\rhead{\thepage} 
\lfoot{\small\scshape EECS 598: Category Theory} 
\cfoot{} 
\rfoot{\footnotesize PS 1} 
\renewcommand{\headrulewidth}{.3pt} 
\renewcommand{\footrulewidth}{.3pt}
\setlength\voffset{-0.25in}
\setlength\textheight{648pt}

%%%%%%%%%%%%%%%%%%%%%%%%%%%%%%%%%%%%%%%%%%%%%%%

%
%Contents of problem set
%

\begin{document}

\newcommand{\skipp}{\textrm{skip}}
\newcommand{\Set}{\textrm{Set}}
\newcommand{\Cat}{\textrm{Cat}}
\newcommand{\Cayley}{\textrm{Cayley}}
\newcommand{\Preds}{\mathcal{P}}
\newcommand{\triple}[3]{\{#1\}{#2}\{#3\}}
\newcommand{\Triple}{\textrm{Triple}}
\newcommand{\Analyse}{\textrm{Analyse}}
\newcommand{\command}{\textrm{command}}

\newcommand{\id}{\textrm{id}}

\title{Problem Set 2}
\date{January 20, 2022}
\maketitle

Homework is due midnight before next Thursday's class.

Note that you may submit your solutions jointly with a partner on
Canvas.

%% assignments will be presented in class. Turn in your homework on
%% Canvas. You may work alone or in groups of two. I highly encourage you
%% to write your solution in LaTeX. The LaTeX packages tikz and tikz-cd
%% are useful for typesetting commutative diagrams, and there are WYSIWYG
%% editors online that produce LaTeX code such as https://q.uiver.app/
%% and https://tikzcd.yichuanshen.de/ .

\thispagestyle{empty}

\begin{problem}{}
  Awodey Section 1.9, Exercise 11, part b\footnote{Part a is
  recommended as well, but will not be graded since the solution is in
  the back of the book}.
\end{problem}

\begin{problem}{}
  %% \begin{enumerate}
  %% \item Show that $K(C)$ is a category with composition given by
  %%   composition in the original category $C$ (what are the identity
  %%   morphisms?).
  %% \item Show that every idempotent splits in $K(C)$
  %% %% \item Show that $K(C)$ has the following UMP.
  %% \end{enumerate}
  
  This problem is inspired by \citet{scott1980relating}.

  Software contracts \citep{ff02} are used for runtime
  assertions in untyped programming languages or when a type system is
  not sophisticated enough to describe the desired invariant.

  A simple\footnote{in practice, a bit too general, but a good
  starting point} model of a contract in a category is an
  \emph{idempotent}. An idempotent in a category $C$ is simply an
  endomorphism:
  \[ c : A \to A \]
  That is equal to its composition with itself
  \[ c \circ c = c \]

  We think of the idempotent $c$ as a (partial) function that coerces
  everything to conform to some property. It is idempotent because
  once something is coerced, it already has the desired property.

  For instance the values of a simple first-order dynamically typed
  language supporting integers and booleans might be modeled by a
  universal set of tagged values:
  \[ D = \{ (0, b) | b \in \{ \textrm{true}, \textrm{false}\}\} \cup \{ (1, n) | n \in \mathbb Z \} \]
  and functions in the language as \emph{partial} functions, with
  undefinedness used to model errors and infinite loops.

  Then a contract $\textrm{int} : D \rightharpoonup D$ for integers
  could be defined as
  \[ \textrm{int}((1,n)) = (1, n) \]
  and undefined on other values. Clearly this is idempotent.

  We would like to think of an idempotent as presenting an object
  itself, the ``invariants'' of the idempotent, i.e., those that
  ``satisfy the contract''. In $\Set$ we could define the invariants
  of an idempotent $c$ on $A$ as $\{ x \in A | c(x) = x \}$. We
  generalize this idea to a general category as follows.

  We say an idempotent $c$ on $A$ is \emph{split} if there exists an
  object $I$ and a \emph{section-retraction} pair
  \[ s : I \to A \]
  \[ r : A \to I \]
  that is, $r \circ s = \id$, that splits the idempotent in that $s
  \circ r = c$.

  Note that for any section/retraction pair $s, r$, the composite $s
  \circ r$ is an idempotent (exercise!). We can then ask if in a
  particular category every idempotent splits. For instance, in $\Set$
  every idempotent splits, using the set of invariants of the
  idempotent. However, clearly not every category has this property.
  
  For any category $C$, we can construct a category $K(C)$, called the
  \emph{Karoubi envelope} (sometimes called the \emph{Cauchy
  completion}) that extends $C$ with the splittings of all
  idempotents.

  Define the Karoubi envelope $K(C)$ as follows.
  \begin{enumerate}
  \item Objects are pairs $(A, c)$ of an object in $C$ and an idempotent $c$ on $A$.
  \item A morphism from $(A, c)$ to $(B, d)$ is a morphism $f : A \to B$ satisfying
    \[ d \circ f \circ c = f \]
  \end{enumerate}

  If we think of an idempotent as a contract, this provides us with a
  category where ``types are contracts''.

  First, complete the definition of the category $K(C)$:
  \begin{enumerate}
  \item Show that composition in $K(C)$ can be given by composition in
    $C$, i.e., that the composition interacts properly with the
    idempotents\footnote{Hint: It may be helpful to prove that the
    condition $d \circ f \circ c = f$ is equivalent to $d \circ f = f
    = f \circ c$ first.}.
  \item Define the identity morphisms and show they are identities
    with respect to composition in $K(C)$.
  \end{enumerate}

  Show that $K(C)$ is the ``idempotent splitting completion'' of $C$:
  \begin{enumerate}
  \item Show that every idempotent in $K(C)$ splits.
  \item Define a functor $\eta : C \to K(C)$.
  \item Show that for any category $D$ where every idempotent splits,
    any functor $F : C \to D$ can be extended to a functor $\hat F :
    K(C) \to D$ satisfying:

    \[ % https://tikzcd.yichuanshen.de/#N4Igdg9gJgpgziAXAbVABwnAlgFyxMJZABgBoBGAXVJADcBDAGwFcYkQBhEAX1PU1z5CKcqWLU6TVuwDSACg4BKHnxAZseAkVEAmCQxZtEIACI8JMKAHN4RUADMAThAC2SMiBwQkOmgenGAGIqDs5uiB5eSKKShuwAOvEwOPQhIE6u0TRRiL6xASCJABb0OAAEwdyU3EA
    \begin{tikzcd}
      & K(C) \arrow[dd, "\hat F"] \\
      C \arrow[rd, "F"] \arrow[ru, "\eta"] &                           \\
      & D                        
    \end{tikzcd} \]

    You do not need to show this functor is unique\footnote{because it
    is only ``unique up to isomorphism'', something we have not yet
    defined}.

    For the logically-inclined, note that constructing $\hat F$ will
    use the axiom of choice.
  \end{enumerate}

\end{problem}

\bibliography{cats}
\end{document}
