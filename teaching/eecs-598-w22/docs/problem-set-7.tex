\documentclass[12pt]{article}

%AMS-TeX packages
\usepackage{amssymb,amsmath,amsthm} 
%geometry (sets margin) and other useful packages
\usepackage[margin=1.25in]{geometry}
\usepackage{tikz}
\usepackage{tikz-cd}
\usepackage{graphicx,ctable,booktabs}
\usepackage{mathpartir,tipa}

\usepackage[sort&compress,square,comma,authoryear]{natbib}
\bibliographystyle{plainnat}

%
%Redefining sections as problems
%
\makeatletter
\newtheorem{lemma}{Lemma}
\newenvironment{problem}{\@startsection
       {section}
       {1}
       {-.2em}
       {-3.5ex plus -1ex minus -.2ex}
       {2.3ex plus .2ex}
       {\pagebreak[3]%forces pagebreak when space is small; use \eject for better results
       \large\bf\noindent{Problem }
       }
       }
       {%\vspace{1ex}\begin{center} \rule{0.3\linewidth}{.3pt}\end{center}}
       \begin{center}\large\bf \ldots\ldots\ldots\end{center}}
\makeatother


%
%Fancy-header package to modify header/page numbering 
%
\usepackage{fancyhdr}
\pagestyle{fancy}
%\addtolength{\headwidth}{\marginparsep} %these change header-rule width
%\addtolength{\headwidth}{\marginparwidth}
\lhead{Problem \thesection}
\chead{} 
\rhead{\thepage} 
\lfoot{\small\scshape EECS 598: Category Theory} 
\cfoot{} 
\rfoot{\footnotesize PS 1} 
\renewcommand{\headrulewidth}{.3pt} 
\renewcommand{\footrulewidth}{.3pt}
\setlength\voffset{-0.25in}
\setlength\textheight{648pt}

%%%%%%%%%%%%%%%%%%%%%%%%%%%%%%%%%%%%%%%%%%%%%%%

%
%Contents of problem set
%

\begin{document}

\newcommand{\skipp}{\textrm{skip}}
\newcommand{\boxprod}{\mathop{\square}}
\newcommand{\Set}{\textrm{Set}}
\newcommand{\Cat}{\textrm{Cat}}
\newcommand{\Ob}{\textrm{Ob}}
\newcommand{\Cayley}{\textrm{Cayley}}
\newcommand{\Preds}{\mathcal{P}}
\newcommand{\triple}[3]{\{#1\}{#2}\{#3\}}
\newcommand{\Triple}{\textrm{Triple}}
\newcommand{\Analyse}{\textrm{Analyse}}
\newcommand{\command}{\textrm{command}}

\newcommand{\id}{\textrm{id}}
\newcommand{\cat}{\mathbb}
\newcommand{\esh}{\text{\textesh}}


\newcommand{\isatype}{\,\,\textrm{type}}
\newcommand{\matchZero}{\textrm{match}_0}
\newcommand{\matchSum}[3]{\textrm{match}_+ {#1}\{{#2}\}\{{#3}\}}

\newcommand{\Yo}{\textrm{Y}}

\title{Problem Set 7}
\date{April 9, 2022}
\maketitle

Homework is due the midnight before class on April 14.

If you haven't yet presented a homework in class, email me to setup
eith an in-class presentation or a presentation outside of class.

\begin{problem}{Examples of Representables}
  The yoneda embedding $\Yo : \cat C \to \Set^{\cat C^o}$ takes an
  object $A \in \cat C$ to the representable presheaf $\lambda X. \cat
  C_1(X,A)$. In a precise sense, all presheaves are constructed by
  gluing together the representable presheaves (Awodey Proposition
  8.11).

  Let's consider what the representable presheaves look
  like in some categories of presheaves that are familiar mathematical
  objects.

  \begin{enumerate}
  \item Let $\mathcal G$ be the category
    \[% https://tikzcd.yichuanshen.de/#N4Igdg9gJgpgziAXAbVABwnAlgFyxMJZABgBpiBdUkANwEMAbAVxiVpAF9T1Nd9CUARnJVajFm1YdRMKAHN4RUADMAThAC2SMiBwQkwkACMYYKEgDMO+s1aIQCLivVbEOvQeomzSALRXqGwl7HBBqBjoTBgAFXjwCNlUsOQALUOkOIA
    \begin{tikzcd}
      v \arrow[r, "s", bend left] \arrow[r, "t"', bend right] & e
    \end{tikzcd}\]
    Then $\Set^{\mathcal G^o}$ is equivalent to the category of
    \emph{multigraphs}, i.e., pairs of a set $V$ of vertices, $E$ of
    edges such that each edge has a source and target vertex.

    Describe for each object $a \in \mathcal G$ what the multigraph
    $\Yo a$ is. Describe for each object $a \in \mathcal G$ and
    multigraph $X$ what a homomorphism $\Yo a \to X$ is.

  \item Let $\mathcal M$ be a one-object category, i.e., a
    monoid. Then $\Set^{\mathcal M^o}$ is equivalent to the category
    of \emph{$\mathcal M$-actions}\footnote{If $\mathcal M$ is a group
    this is equivalently a group action}. We write the identity
    element of $\mathcal M$ as $\epsilon$ and the multiplication as $m
    \cdot m'$.

    An $\mathcal M$-action is a set
    $X$ equipped with a ``multiplication'' operation $\times :
    \mathcal M \times X \to X$ satisfying (for any $m, m_1,m_2 \in
    \mathcal M$, and $x \in X$):
    \[ \epsilon \cdot x = x \]
    \[ m_1 \times (m_2 \times x) = (m_1 \cdot m_2) \times x \]

    An \emph{equivariant map} $\phi : X \to Y$ between $\mathcal
    M$-actions is a function on the underlying sets satisfying
    \[ \phi(m \times x) = m \times \phi(x) \]

    Let $*$ be the object of $\mathcal M$ viewed as a one object
    category. Describe what the $\mathcal M$-action $\Yo *$ is, and
    describe what an equivariant map $\Yo * \to X$ is.

  \item Let $I$ be a set, viewed as a category with only identity
    arrows. Then $\Set^{I^o} = \Set^I$ is equivalent to the category
    of \emph{families indexed by $I$}, i.e., $\{X_i \in \Set \}_{i \in
      I}$. A morphism of families $\{ X_i \}_{i \in I} \to \{Y_i\}_{i
      \in I}$ is a family $\{f_i : X_i \to Y_i\}_{i \in I}$.

    For each $i\in I$, describe what family $\Yo i$ is and what the
    morphisms $\Yo i \to \{X_i \}_{i\in I}$.
  \end{enumerate}

  \textbf{Hint: to describe the morphisms, use the Yoneda lemma!}
\end{problem}

\begin{problem}{A String of Adjoints}
  Awodey Chapter 9, Problem 7.
\end{problem}

\thispagestyle{empty}

\bibliography{cats}
\end{document}
